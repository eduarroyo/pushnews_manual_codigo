\documentclass[a4paper,12pt,twoside,final]{scrbook}
\usepackage[utf8]{inputenc}
\usepackage[spanish,es-nodecimaldot]{babel} % International, rec. by RAE: http://www.tex-tipografia.com/marca_decimal.html

\usepackage{booktabs}
\usepackage[hidelinks]{hyperref}

% COLOR
\usepackage[usenames,dvipsnames,svgnames,table]{xcolor}
% Colores para estilo Proyecto Docente (tonos naranjas)
\definecolor{lightback}{HTML}{F4E0BF}
     \definecolor{back}{HTML}{F3C591}
\definecolor{lightline}{HTML}{FCAF5F}
     \definecolor{line}{HTML}{ED7900}
% Colores para portada
\definecolor{epsc:oscuro}{HTML}{280091}
 \definecolor{epsc:medio}{HTML}{4C5CC5}
 \definecolor{epsc:claro}{HTML}{3FCFD5}
 \definecolor{epsc:verde}{HTML}{00B299}

\usepackage{tikz} % used in cover to place images

\usepackage{datetime} % allow formal date format
% "Month, YEAR" date format, in spanish with the month uppercased not interfering other dates
\newcommand\Monthname[1][EMPTY]{
  \ifnum1=#1Enero\else
  \ifnum2=#1Febrero\else
  \ifnum3=#1Marzo\else
  \ifnum4=#1Abril\else
  \ifnum5=#1Mayo\else
  \ifnum6=#1Junio\else
  \ifnum7=#1Julio\else
  \ifnum8=#1Agosto\else
  \ifnum9=#1Septiembre\else
  \ifnum10=#1Octubre\else
  \ifnum11=#1Noviembre\else
  \ifnum12=#1Diciembre\else
  \fi\fi\fi\fi\fi\fi\fi\fi\fi\fi\fi\fi
}
\newdateformat{monthyeardate}{%
  \Monthname[\THEMONTH], \THEYEAR}


% FONTs
\usepackage[T1]{fontenc}
\usepackage{textcomp}           % Needed for new symbols like € ?

\usepackage[scaled]{berasans} % Font for the cover similar to Vera 33
%\renewcommand*\familydefault{\sfdefault}  %% To use as the base font of the document is to be sans serif


% PAGE STYLE
\usepackage[twoside,bindingoffset=0mm,headheight=16pt,margin=25mm]{geometry} %,verbose,showframe
%\usepackage{fancyhdr} % Encabezados
%\pagestyle{fancy}
% XXX Evitar el binding en la portada pero no en el resto del documento

% TABLES AND FIGURES
\usepackage{graphicx}
\graphicspath{ {img/} }

% SUBRALLADOS
\usepackage[normalem]{ulem}



\begin{document}
%\frontmatter

%------------- Cover --------------
\thispagestyle{empty}

% Backgroud images
\begin{tikzpicture}[remember picture, overlay]
  % Top
  \node [anchor=north east, inner sep=0pt]  at (current page.north east)
     {\includegraphics[height=6cm]{topRightCorner.pdf}};
  % Bottom
  \node [anchor=south west, inner sep=0pt]  at (current page.south west)
     {\includegraphics[height=6cm]{bottomLeftCorner.pdf}};
  \node (uco) [anchor=south east, inner sep=0pt, xshift=-10mm, yshift=10mm]  at (current page.south east)
        {\includegraphics[height=2cm]{uco_debajo.pdf}};
  \node [anchor=south east, inner sep=0pt, xshift=-10mm]  at (uco.south west)
% Uncomment the chosen logo and comment the others:
        {\includegraphics[height=2cm]{emblema-ing-informatica.pdf}};
%        {\includegraphics[height=2cm]{emblema-ing-industrial.pdf}};
%        {\includegraphics[height=2cm]{emblema-ing-tec-industrial.pdf}};
\end{tikzpicture}


\begin{center}
\fontfamily{\sfdefault}\selectfont
\vspace*{2cm}

\vfill
\vfill
\includegraphics[width=12.5cm]{LogotipoEPSC.pdf}
\vfill
\vfill

\large\textbf{\color{epsc:medio}
  TRABAJO FIN DE GRADO EN INGENIERÍA INFORMÁTICA
}
\vfill

\Large\textbf{\color{epsc:verde}
 2 - MANUAL DE CÓDIGO
}
\vfill
\vfill

\Huge\textbf{\color{epsc:oscuro}
  Plataforma web de gestión de comunicados
}
\vfill
\vfill

\large{\color{epsc:oscuro}Autor}\\
\textbf{\color{epsc:medio}{ Eduardo Arroyo Ramírez }}
\vfill

\large{\color{epsc:oscuro} Director }\\
\textbf{\color{epsc:medio} Domingo Ortiz Boyer}
\vfill



\textbf{\color{epsc:verde} \monthyeardate\today}
\vfill
\vfill
\vspace{2.7cm}
\end{center}

%-------------------------------------------------------------------------------
\clearpage

\thispagestyle{empty}
\pagecolor{white}

\cleardoublepage
%-------------------------------------------------------------------------------
\frontmatter
\tableofcontents
\listoffigures
\listoftables

\cleardoublepage
\mainmatter
\include{01-introduccion}
% \cleardoublepage
% \include{02-definicion-del-problema}
% \cleardoublepage
% \include{03-antecedentes}
% \cleardoublepage
% \include{04-objetivos}
% \cleardoublepage
% \include{05-recursos}
% \cleardoublepage
% \include{06-aplicacion-web}
% \cleardoublepage
% \include{07-servicio-web}
% \cleardoublepage
% \include{08-aplicacion-movil}
% \cleardoublepage
% \include{09-seguridad}
% \cleardoublepage
% \include{10-futuras-mejoras}
% \cleardoublepage
% \include{11-conclusiones}

% BIBLIOGRAFÍA -----------------------------------------------------------------
\bibliographystyle{ieeetr}
\renewcommand{\refname}{Bibliografía}
\bibliography{00-manual-codigo}
\addcontentsline{toc}{part}{Bibliografía}
\clearpage
% ------------------------------------------------------------------------------

\end{document}